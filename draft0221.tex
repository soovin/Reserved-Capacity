\documentclass{article}

\usepackage{amsmath}
\usepackage{amsfonts}
\usepackage{amssymb}
\usepackage{enumerate}
%\usepackage{verbatim}
\usepackage[pdftex]{graphicx}
\usepackage{todonotes}
\graphicspath{ {images/}}

\usepackage[square,numbers]{natbib}
\bibliographystyle{abbrvnat}

%\addtolength{\hoffset}{-1.0cm} 
%\addtolength{\textwidth}{1cm}
%\usepackage{blindtext}
\usepackage[utf8]{inputenc}

% some margins work
\topmargin 0.0cm
\oddsidemargin 0.2cm
\textwidth 16cm 
\textheight 21cm
\footskip 1.0cm

\title{An Expected Coverage Model with a Cutoff Priority Queue}
\author{Soovin Yoon}

\begin{document}
\maketitle


%\tableofcontents



\section{Introduction}
\todo[inline]{Briefly suggest the context and importance of the problem of reserving ambulances in urban EMS, what this paper proposes and what contribution it makes.}

The paper makes following contributions. First, we provide a queuing model that can be utilized to

Second,the impact of reserving capacity in urban emergency medical services is investigated in various metrics. 

\section{Literature Review}
\subsection{Queuing Literature}
\subsubsection{Hypercube Literature}
\subsubsection{Cutoff Queuing Literature}
\subsection{Location Literatures with Priorities}


%%%%%%%%%%%%%%%%%%%%%%%%%%%%%%%%%%%%%%%%%%%%%%%%%%%%%%%%%%%%%%%%%%%%%%%%%%%%%%%%
% Hypercube model
%%%%%%%%%%%%%%%%%%%%%%%%%%%%%%%%%%%%%%%%%%%%%%%%%%%%%%%%%%%%%%%%%%%%%%%%%%%%%%%%

\section{Approximate Hypercube Model for Cutoff Priority Queue}

%\todo[inline]{Also provide formulas for infinity-line queue case}

In this section, we derive a Hypercube model approximation for cutoff priority queue. The Hypercube model functions as standalone program for analysis of characteristics of cutoff priority queue. Also, the model iteratively interacts with MILP model described in the next section to assist decision-making for deployment and dispatch of ambulances under the uncertainty in their availability. The procedure introduced in this section extends the iterative procedure given by Larson(1975) for approximating equilibrium behavior and performance measures for the Hypercube model. 

A system with $s$ distinguishable servers and high(low) priority calls from $\|J\|$ locations is considered. High(low) priority calls arrive according to a Poisson process at a rate $\lambda^H(\lambda^L)$. A cutoff level of $s_1$ is given for admission of low priority customers. This means that low priority calls are placed in a queue whenever $s_1$ or more servers are busy. High priority calls are immediately served as long as there are free servers. They are either put in a queue or lost if all servers are busy. This implies that the server cutoff level for high priority customers is set to $s$ by convention(Larson 1986). The service is assumed to be non-preemptive.

We consider the two type of model that differs by the queue capacity. Under the assumption of $0-$line capacity system (M/M/N/$0$), high priority calls arriving when all servers are busy or low priority calls arriving when there are more than or equal to $s_1$ busy servers are "lost to the system", as the queue has zero capacity. Lost calls are considered as being served by external resources. It is common in the literature to assume the system with no queue(Budge et al. 2009). Also, this is consistent with the emergency medical service reality.

On the other hand, if the queue is assumed to have $\infty-$line capacity(M/N/M/$\infty$), calls are put in a queue when they cannot be served immediately. They are served later in a first-come-first-served manner. As the queue is infinitely large, backlogged calls are never lost as they are eventually served in the long run. Here we show the derivation for both queue size assumptions. 



\begin{table}
\centering
\begin{tabular}{c l}
\hline
\hline
\textbf{symbol} & \textbf{description} \\
\hline
 $i \in J$	& set of customer locations \\
 $i \in I^{open}$ & set of open stations \\
 $p \in \{H,L\}$ & set of call priorities \\
 $ s $ 	& total number of servers \\
 $ s_1 $& cutoff level for low priority calls\\
 $\rho$ & server utilization \\
 $ r_i$ & busy probability of server at station $i$ \\
 $ r $ 	& system-wide server busy probability \\
 $P_i$ 	& steady-state probability \\
 $q_k$ 	& correction factor \\
 $f_{jk}$ & probability of dispatching $k$th preferred server to call location $j$ \\ 
 $b_{jk}$ & the station where the server $k$th preferred by call location $j$ is located \\
 $\lambda_{j}^p$ & arrival rate of priority $p$ call from location $j$\\
 $\lambda^H(\lambda^L)$ & arrival rate of high(low) priority calls\\
 $\lambda$ & system-wide call arrival rate\\
 $\tau_{ij}$ & mean service time for calls from location $j$ served by a server from station $i$ \\
 $\tau$ & system-wide mean service time\\
\hline
\end{tabular}
\caption{Summary of Symbols}
\label{table:symbol}
\end{table}


Table \ref{table:symbol} summarizes symbols used frequently. The hypercube queuing model requires deploying and dispatching policy as inputs, which can be generated either externally(for example, send-the-closest) or by solving the MILP in the next section. There are other parameters needed such as service times and arrival rates, and these can be collected from the data. The outputs of the model are correction factors $Q_k$ for $k=1,\cdots,s$ and server busy probabilities $r_i$ for $i=1,\cdots,s$. Then dispatch probability $f_{jk}$, the probability that a priority $p$ call from location $j$ is served by its $k$th preferred server, can be approximated as
\begin{equation} \label{dispatchprobability}
f^p_{jk} = Q^p_{k-1} \Pi_{l=1}^{k-1} r_{b^p_{jl}}(1-r_{b^p_{jk}}).
\end{equation}
The correction factors $Q_k$ is needed to approximately correct for the unrealistic assumption of independent servers. 


\begin{description}
\item[STEP 0]
Given:

$\lambda_j^p, \tau_{ij}$ from inputs, open stations $I^{open}$ and preference list $b_{jk}$

Initialize:

\begin{align}
\tau^0=\sum_{j\in J} \sum_{i \in I^{open}} \sum_{p\in \{H,L\}} \tau_{ij} \frac{\lambda^p_j}{\lambda^p}
\end{align}
\begin{align}
\rho=\frac{\lambda \tau}{s}
\end{align}

\item[STEP 1] Update steady-state probabilities $P_0$ and $P_s$.

For $0-$line capaticy queue, use
\begin{align}
P_i &= P_0 \frac{s^i\rho^i}{i!}, \quad 0 \leq i \leq s_1\\ \nonumber
P_i &= P_0 \frac{s^i\rho^i}{i!} \big(\frac{\lambda^H}{\lambda}\big)^{i-s1}, \quad s1+1 \leq i \leq s
\end{align}
where
\begin{align}
P_0 = \big( \sum_{i=0}^{i=s1} \frac{s^i\rho^i}{i!} + \sum_{i=s1+1}^s  \frac{s^i\rho^i}{i!} \big(\frac{\lambda^H}{\lambda}\big)^{i-s1} \big)^{-1}.
\end{align}

For $\infty-$line capacity queue, denoting $\rho_H=\lambda \tau P_H, \rho_L = \lambda \tau P_L, \rho = \rho_H+\rho_L$, use
\begin{align}
P_n & = P_0 \frac{\rho^s}{s!} \quad 0 \leq n \leq s_1-1\\
      & = P_0(\rho^{s_1} \rho_H^{n-s_1}/n!)s_1[s_1-\rho_L S_L(s_1,s)]  \quad s_1 \leq n \leq s
\end{align}
where
\begin{align}
P_0 = \{\sum_{n=0}^{s_1-1} \rho^n / n! + (\rho^{s_1}/(s_1-1)!)SL(s_1,s)/[s_1-\rho_L S_L(s_1,s)])   \}^{-1},
\end{align}
\begin{align}
S_L(k,s) = \sum_{i=k}^s \rho_H^{i-k} k!/i!.
\end{align}

\item[STEP 2] Update correction factors $Q_k^H$ and $Q_k^L$ by
\begin{align}
 Q_k^H  & = \dfrac{ \sum_{j=k}^{s-1} \dfrac{ (s-k-1)! (s-j) j!}{(j-k)!s!} P_k }
      { w^k \rho^k  \big( 1-(w\rho \big) },
\end{align}
and
\begin{align}
 Q_k^L  & = \dfrac{ \sum_{j=k}^{s_1-1} \dfrac{ (s-k-1)! (s-j) j!}{(j-k)!s!} P_k }
      { w^k \rho^k  \big( 1-w\rho \big) }.
\end{align}

For $0-$line capacity queue, use
\begin{align}
w = (1-P_s \dfrac{\lambda_H}{\lambda}-\sum_{i=s_1}^s P_{i} \dfrac{\lambda_L}{\lambda}) 
\end{align}
and $w=1$ for $\infty-$line capacity queue.


\item[STEP 3] Update server busy probabilities $r_i$ as
\begin{align}
r_i = \frac{V_i}{V_i+1}.
\end{align}
where
\begin{align}
V_i = \sum_{i \in I^{Open}} \sum_{k \in K} \sum_{p \in \{H,L\}} \lambda^p_j \tau_{b^p_{jk}j} q^p_k (\Pi^{k-1}_{l=1} r_{b^p_{jl}})
\end{align}


\item[STEP 4] Update system-wide service time $\tau$ as
\begin{align}
\tau = \sum_{i \in I^{open}} \sum_{j \in J} \tau_{ij} \frac{\lambda^p_j}{\lambda}\frac{f^p_{j,a^p_{ij}}}{\sum_{k \in K} f^p_{jk}}
\end{align}

where $f^H_{jk}$ and $f^L_{jk}$ follows
\begin{align}
f^p_{jk} = Q^p_{k-1} (1-r_{b^p_{jk}}) (\Pi^{k-1}_{l=1} r_{b^p_{jl}})
\end{align}

\item[STEP 5] Update $\rho=\frac{\lambda \tau}{s}$ and $r = w \rho$.

Normalize $r_i$ by
\begin{align}
r_i \leftarrow \frac{r_i}{\sum_i r_i/s} r
\end{align}
Check if max change in $r_i$ is below the predetermined threshold. \\
If yes, terminate with the return value $r$ and $Q_k^p$. If no, go back to step 1.

\end{description}

First, we need to derive steady-state probabilities in cutoff priority queue. For $0-$line queue, Figure \ref{fig:transitiondiagram} shows the corresponding transition diagram.

\begin{figure}
\centering
\includegraphics[scale=0.5]{transitiondiagram}
\caption{Transition Diagram for $0-$line Capacity Queue}
\label{fig:transitiondiagram}
\end{figure}

Therefore, by solving the following balance equations, we gain the desired probability.
\begin{align*} P_1= \lambda \tau P_0 =\rho s P_0, \cdots, P_i = \dfrac{\rho^i s^i}{i!} P_0 \quad \quad i\leq s_1,
\end{align*}
\begin{align*}
P_{s_1+1} = \lambda_H \dfrac{\tau}{ s_1+1} P_{s_1}=\dfrac{\rho s}{s_1+1} \frac{\lambda_H}{\lambda} P_{s_1}=\dfrac{\rho^{s_1}}{s_1+1!} s^{s_1} \frac{\lambda_H}{\lambda} P_{0},
\end{align*} 
\begin{align*}
\cdots, P_i = \dfrac{\rho^i s^i}{i!} (\dfrac{\lambda_H}{\lambda})^{i-s_1} P_0 \quad \quad i > s_1
\end{align*} 

In order to derive the steady-state probabilities formulation for the $\infty-$line queue, SCQL model in Taylor and Templeton(1980) was referred.

Next, correction factors for low priority in cutoff priority queue is derived. This follows Larson(1975)'s procedure to derive an expression for $P \{ B_1b_2 \cdots B_jF_{j+1}\}$ (p852) closely, punctuated by some changes due to cutoff assumption. 

Let $B_j \equiv $ event that $j$th selected is busy (not available), $F_j \equiv B_j^c =$ event that $j$th server selected is free (or available), $S_k \equiv $  state of the queuing system indicating that exactly $k$ servers are busy, and $P \{ B_1B_2 \cdots B_j F_{j+1}\} \equiv $ probability that the first free server is the $j+1$st server selected. 

Under the cutoff priority queue assumption, if the arriving call is low priority, we have 
\begin{align*}P \{ B_1B_2 \cdots B_j F_{j+1}\}=P \{ B_1b_2 \cdots B_jF_{j+1}|S_k,k<s_1\}
\end{align*}
because we do not select a server to dispatch to a low priority call, if the number of busy server is above the cutoff.

Therefore we wish to derive an expression for $P \{ B_1b_2 \cdots B_jF_{j+1}|S_k,k<s_1\}$ that will motivate an approximation procedure for the model in which servers are not identical. (Note: In the hypercube model under a fixed-preference dispatching policy, the dispatcher always assigns the most preferred available server. Therefore the desired probability is the probability that the first $j$ preferred servers are busy and the $j+1$st is free.)

By laws of conditional probability we have
\begin{align*} 
P \{ B_1b_2 \cdots B_jF_{j+1}|S_k,k<s_1\} = \sum_{k=0}^{k=s_1-1} P \{ B_1b_2 \cdots B_jF_{j+1}|S_k\} P_k
\end{align*}
Where
\begin{align*}
P \{ B_1b_2 \cdots B_jF_{j+1}|S_k\} = P\{F_{j+1}|B_1B_2 \cdots B_j S_k\} P\{B_j|B_1B_2 \cdots B_{j-1}S_k \} \cdots P\{ B_1|S_k\}.
\end{align*}
Then we have
\begin{align*}
P\{B_i|B_1B_2 \cdots B_{i-1}S_k \} &= \dfrac{k-(i-1)}{s-(i-1)} \quad \quad i=1,2, \cdots, k+1 \\
P\{F_{j+1}|B_1B_2 \cdots B_j S_k\} & = \dfrac{s-k}{s-j} \quad \quad  j =0,1,\cdots, k
\end{align*}

Combining above results we have the desired probability
\begin{align*}
P \{ B_1b_2 \cdots B_jF_{j+1}|S_k,k<s_1\} &= \sum_{k=j}^{s_1-1} \dfrac{k}{s} \dfrac{k-1}{s-1}  \cdots \dfrac{k-(j-1)}{s-(j-1)} \frac{s-k}{s-j} P_k\\
 & = \sum_{k=j}^{s_1-1} \dfrac{ (s-j-1)! (s-k) k!}{(k-j)!s!} P_k
\end{align*}
with $P_k$ derived in the note above, or, 
\[ P \{ B_1 B_2 \cdots B_j F_{j+1}|S_k,k<s_1\} = Q_j^L r^j (1-r),\]
where
\begin{align*}
 Q_j^L  & = \dfrac{ \sum_{k=j}^{s_1-1} \dfrac{ (s-j-1)! (s-k) k!}{(k-j)!s!} P_j }
      { (1-P_s \dfrac{\lambda_H}{\lambda}-\sum_{i=s_1}^s P_{i} \dfrac{\lambda_L}{\lambda})^j \rho^j  \big( 1-(1-P_s \dfrac{\lambda_H}{\lambda}-\sum_{i=s_1}^s P_{i} \dfrac{\lambda_L}{\lambda})\rho \big) }.
\end{align*}

Lastly, the derivation of $Q_j^H$ is just analogous to above $s_1$ replaced by $s$.




%%%%%%%%%%%%%%%%%%%%%%%%%%%%%%%%%%%%%%%%%%%%%%%%%%%%%%%%%%%%%%%%%%%%%%%%%%%%%%%%
% MILP model
%%%%%%%%%%%%%%%%%%%%%%%%%%%%%%%%%%%%%%%%%%%%%%%%%%%%%%%%%%%%%%%%%%%%%%%%%%%%%%%%

\section{Maximum Expected Coverage Model}
In this section, a mixed integer linear program model that generates deploying and dispatching solutions that maximize expected coverage over the region is introduced. A spatial service system with $s$ servers, a set of potential stations $I$ and a set of call locations $J$ where calls arrive in a Poisson manner is considered. Exactly $s$ stations are selected as open stations and one server is located for each open station. Also, each call location $j$ is assigned a preference list $b^H_{jk}(b^L_{jk})$, which is an ordered list of open stations for high(low) priority calls.

Define the binary variable $y_i=1(0)$ if a server is located at station $i$. This represents deployment decisions that defines the set of open stations. Also define the set of Bernoulli variables $x^p_{ijk}=1(0)$ if it is(not) the $k$th preferred station for a priority $p$ call from location $j$. This captures the dispatch policy. 




The program is formally stated below. 

\begin{equation} \label{eq:obj}
\max \sum_{i=1}^s \sum_{j \in J} \sum_{k=1}^s c_{ijk} x^H_{ijk}
\end{equation}
subject to
\begin{equation} \label{eq:y}
\sum_{i \in I} y_i =s, \quad y_i \in \{ 0,1\}
\end{equation}
\begin{equation} \label{eq:xy}
\sum_{k=1}^s x_{ijk}^p  = y_i \qquad \forall i,j,p
\end{equation}
\begin{equation} \label{eq:x}
\sum_{i \in I} x_{ijk}^p = 1 \qquad \forall j,k,p
\end{equation}
\begin{equation} \label{eq:bal1}
\sum_{p,j,k} \lambda_{j}^{p} q_{k} (1-r)r^{k-1} \tau_{ij} x_{ijk}^p \leq (r+\delta) y_i
\end{equation}
\begin{equation} \label{eq:bal2}
\sum_{p,j,k} \lambda_{j}^{p} q_{k} (1-r)r^{k-1} \tau_{ij} x_{ijk}^p \geq (r-\delta) y_i
\end{equation}
\begin{equation} \label{eq:contiguity}
x_{ij'1}^p \geq x_{ij1}^p, \quad \forall p \in \{H,L\},  j \in J,i \in I, j' \in N_{ij}
\end{equation}
\begin{equation} \label{eq:coefficient}
c_{ijk}=q_{k-1}(1-r) r^{k-1} R_{ij} \frac{\lambda_j^H}{\lambda^H} 
\end{equation}

The objective function in \eqref{eq:obj} represents expected coverage over high priority calls. A station is considered to be opened if a server is located, and there are exactly $s$ open stations which is guaranteed by \eqref{eq:y}.
\eqref{eq:xy} ensures that a server from station $i$ is dispatched only when the station $i$ is open. Also, \eqref{eq:x} enforces the call location to have an ordered, non-overlapping preference list. 

\eqref{eq:contiguity} refers to contiguity constraints, which enforces the program to assign the first priority in a geographically reasonable sense. When the first preferred station for high priority calls from location $j$ is station $i$, and there is a call location $j'$ is a neighborhood of $j$ and closer to $i$ than $j$, then constraint set \eqref{eq:contiguity} enforces $i$ to be the first preferred station for high priority call from location $j$.

Finally, \eqref{eq:coefficient} represents explicit considerations for uncertainty in the ambulance unavailability and the travel time. Ambulance unavailability is captured by introducing $r$, the system-wide busy probability. The use of this common busy probability $r$ implies that it is assumed that the busy probability is the same for all ambulances, and this is reasonable as the load balancing is guaranteed by \eqref{eq:bal1} and \eqref{eq:bal2}. Correction factor $q_k$, which is the output from previous Hypercube model, yields approximation to actual queuing probabilities. Non-deterministic travel times is also implemented by adopting a parameter $R_{ij}$, which represents the probability that an ambulance located at station $i$ may reach the call location $j$ within the time limit. $R_{ij}$ is naturally a decreasing function of the distance between the station $i$ and location $j$. Putting the parameters together, $q_{k-1}(1-r) r^{k-1} R_{ij}$ captures the coverage of the server at station $i$ over high priority calls from location $j$ when $i$ is the $k$th preferred server by high priority call at $j$. And then this value is weighted by its relative call volume proportion $\lambda_j^H/\lambda^H$ to shape the coefficient $c_{ijk}$, so that the objective function represents expected coverage over the entire region.

Once the MILP problem is solved, the preference list $b^p_{jk}$ is created from the optimal value of $x^p_{ijk}$ accordingly; for all $j \in J$, $k=1,\cdots,s$, $i\in I$, $p \in \{H,L\}$, if $x^p_{ijk}=1$, then $b_{jk}=i$.

The system-wide server busy probability $r$ is used as a parameter in objective coefficients as well as load balancing constraints. However, this is an endogeneous value in that the decision of deployment and dispatching itself affects the busy probability of servers. Nevertheless, in pursuit of problem tractablility, we want to avoid expressing $r$ as a function of decision variable $y_i$ and $x^p_{ijk}$, because doing so would make the mixed integer program nonlinear. 

In order to overcome this complication, an iterative scheme is introduced. As mentioned in the previous section, the Hypercube model is utilized to generate the correction factors $q_k$ and server busy probabilities $r$ used as inputs for MILP model. Then the solutions from the MILP model become the new inputs for the Hypercube model to update the value of correction factors and server busy probabilities. Again, those outputs are utilized as inputs for the MILP model. This iterative procedure continues until outputs converges so that the change in server busy probability drops under the threshold.

Table \eqref{table:steps} details the steps in iterative Algorithm that comprehensively describes the iterative use of Hypercube model and MILP model. 


\begin{table}
\noindent\makebox[\linewidth]{\rule{16cm}{0.4pt}}
\begin{description}
\item[STEP 0] Initialize:\\
Generate the initial preference lists $b^p_{jk}$ from the \textit{send-the-closest-available-server} rule. \\
Set correction factors $q_k=1$ for all $k=1,\cdots,s$, server busy probability $r=\rho=\dfrac{\lambda \tau}{s}$, where system-wide mean service time $\tau$ set as $\tau= \lambda^{-1} \sum_i \sum_j  \sum_p \lambda^p_j \tau_{ij}$.  
\item[STEP 1] Solve the MILP model with inputs $r,r_i,q_k$ to update preference lists $b^p_{jk}$ and set of open stations $I^{open}$.
\item[STEP 2] Solve the Hypercube model with inputs $b^p_{jk}$ and $I^{open}$ to update server busy probabilities $r$,$r_i$ and correction factors $q_k$.
\item[STEP 3] Check termination criteria. Stop if (1) max change in $r_i$ is less than the imbalance threshold or if (2) MILP was infeasible. 

If both termination criteria are not satisfied, go back to \textbf{STEP 1}.
\end{description}
\noindent\makebox[\linewidth]{\rule{16cm}{0.4pt}}

\caption{Iterative Algorithm}
\label{table:steps}

\end{table}




%%%%%%%%%%%%%%%%%%%%%%%%%%%%%%%%%%%%%%%%%%%%%%%%%%%%%%%%%%%%%%%%%%%%%%%%%%%%%%%%
% Results
%%%%%%%%%%%%%%%%%%%%%%%%%%%%%%%%%%%%%%%%%%%%%%%%%%%%%%%%%%%%%%%%%%%%%%%%%%%%%%%%

\section{Computational Results}
This section reports 
Throughout the section, simulation results are also paralleled to confirm the validity of cutoff model.

% describe data used

% describe the code used

\subsection{Queuing perspective}

Table show the dispatch probabilities under different cutoff numbers. 

\begin{table}
\centering
$s_1=s-2$\\
\begin{tabular}{l l l l l}
\hline
\hline
k& \multicolumn{2}{c}{Hypercube} & \multicolumn{2}{c}{Simulation} \\
\hline
 & H      &     L  & H      & L      \\
 \hline
1& 0.7025 & 0.6365 & 0.6877 & 0.6236 \\
2& 0.2016 & 0.1503 & 0.1968 & 0.1513 \\
3& 0.0642 & 0.0278 & 0.0726 & 0.0367 \\
4& 0.0213 & 0      & 0.0283 & 0      \\
5& 0.0060 & 0      & 0.0097 & 0      \\
\hline
\end{tabular}

$s_1=s$  :\\
\begin{tabular}{l l l l l}
\hline
\hline
k& \multicolumn{2}{c}{Hypercube} & \multicolumn{2}{c}{Simulation} \\
\hline
 & H      &     L  & H      & L      \\
 \hline
1& 0.6749 & 0.6749 & 0.6601 & 0.6619  \\
2& 0.1988 & 0.1988 & 0.1950 & 0.1932 \\
3& 0.0678 & 0.0678 & 0.0756 & 0.0750 \\
4& 0.0268 & 0.0268 & 0.0311 & 0.0316      \\
5& 0.0012 & 0.0012 & 0.0018 & 0.0176      \\
\hline
\end{tabular}
\label{table:dispatchprob}
\caption{Dispatch Probabilities for Different Cutoff Numbers}
\end{table}


Although the Hypercube model is an approximate model, the discrepancy between the analytical Hypercube model results and simulation results are relatively small enough to conclude that the Hypercube model is valid.

\subsection{Coverage Improvement}

The coverage from the MILP model and simulation results are close enough.



%try \cite{latexcompanion}
%\bibliography{draft0221}



\end{document}